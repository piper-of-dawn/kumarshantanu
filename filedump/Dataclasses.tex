\documentclass[11pt]{article}

    \usepackage[breakable]{tcolorbox}
    \usepackage{parskip} % Stop auto-indenting (to mimic markdown behaviour)
    

    % Basic figure setup, for now with no caption control since it's done
    % automatically by Pandoc (which extracts ![](path) syntax from Markdown).
    \usepackage{graphicx}
    % Maintain compatibility with old templates. Remove in nbconvert 6.0
    \let\Oldincludegraphics\includegraphics
    % Ensure that by default, figures have no caption (until we provide a
    % proper Figure object with a Caption API and a way to capture that
    % in the conversion process - todo).
    \usepackage{caption}
    \DeclareCaptionFormat{nocaption}{}
    \captionsetup{format=nocaption,aboveskip=0pt,belowskip=0pt}

    \usepackage{float}
    \floatplacement{figure}{H} % forces figures to be placed at the correct location
    \usepackage{xcolor} % Allow colors to be defined
    \usepackage{enumerate} % Needed for markdown enumerations to work
    \usepackage{geometry} % Used to adjust the document margins
    \usepackage{amsmath} % Equations
    \usepackage{amssymb} % Equations
    \usepackage{textcomp} % defines textquotesingle
    % Hack from http://tex.stackexchange.com/a/47451/13684:
    \AtBeginDocument{%
        \def\PYZsq{\textquotesingle}% Upright quotes in Pygmentized code
    }
    \usepackage{upquote} % Upright quotes for verbatim code
    \usepackage{eurosym} % defines \euro

    \usepackage{iftex}
    \ifPDFTeX
        \usepackage[T1]{fontenc}
        \IfFileExists{alphabeta.sty}{
              \usepackage{alphabeta}
          }{
              \usepackage[mathletters]{ucs}
              \usepackage[utf8x]{inputenc}
          }
    \else
        \usepackage{fontspec}
        \usepackage{unicode-math}
    \fi

    \usepackage{fancyvrb} % verbatim replacement that allows latex
    \usepackage{grffile} % extends the file name processing of package graphics 
                         % to support a larger range
    \makeatletter % fix for old versions of grffile with XeLaTeX
    \@ifpackagelater{grffile}{2019/11/01}
    {
      % Do nothing on new versions
    }
    {
      \def\Gread@@xetex#1{%
        \IfFileExists{"\Gin@base".bb}%
        {\Gread@eps{\Gin@base.bb}}%
        {\Gread@@xetex@aux#1}%
      }
    }
    \makeatother
    \usepackage[Export]{adjustbox} % Used to constrain images to a maximum size
    \adjustboxset{max size={0.9\linewidth}{0.9\paperheight}}

    % The hyperref package gives us a pdf with properly built
    % internal navigation ('pdf bookmarks' for the table of contents,
    % internal cross-reference links, web links for URLs, etc.)
    \usepackage{hyperref}
    % The default LaTeX title has an obnoxious amount of whitespace. By default,
    % titling removes some of it. It also provides customization options.
    \usepackage{titling}
    \usepackage{longtable} % longtable support required by pandoc >1.10
    \usepackage{booktabs}  % table support for pandoc > 1.12.2
    \usepackage{array}     % table support for pandoc >= 2.11.3
    \usepackage{calc}      % table minipage width calculation for pandoc >= 2.11.1
    \usepackage[inline]{enumitem} % IRkernel/repr support (it uses the enumerate* environment)
    \usepackage[normalem]{ulem} % ulem is needed to support strikethroughs (\sout)
                                % normalem makes italics be italics, not underlines
    \usepackage{mathrsfs}
    

    
    % Colors for the hyperref package
    \definecolor{urlcolor}{rgb}{0,.145,.698}
    \definecolor{linkcolor}{rgb}{.71,0.21,0.01}
    \definecolor{citecolor}{rgb}{.12,.54,.11}

    % ANSI colors
    \definecolor{ansi-black}{HTML}{3E424D}
    \definecolor{ansi-black-intense}{HTML}{282C36}
    \definecolor{ansi-red}{HTML}{E75C58}
    \definecolor{ansi-red-intense}{HTML}{B22B31}
    \definecolor{ansi-green}{HTML}{00A250}
    \definecolor{ansi-green-intense}{HTML}{007427}
    \definecolor{ansi-yellow}{HTML}{DDB62B}
    \definecolor{ansi-yellow-intense}{HTML}{B27D12}
    \definecolor{ansi-blue}{HTML}{208FFB}
    \definecolor{ansi-blue-intense}{HTML}{0065CA}
    \definecolor{ansi-magenta}{HTML}{D160C4}
    \definecolor{ansi-magenta-intense}{HTML}{A03196}
    \definecolor{ansi-cyan}{HTML}{60C6C8}
    \definecolor{ansi-cyan-intense}{HTML}{258F8F}
    \definecolor{ansi-white}{HTML}{C5C1B4}
    \definecolor{ansi-white-intense}{HTML}{A1A6B2}
    \definecolor{ansi-default-inverse-fg}{HTML}{FFFFFF}
    \definecolor{ansi-default-inverse-bg}{HTML}{000000}

    % common color for the border for error outputs.
    \definecolor{outerrorbackground}{HTML}{FFDFDF}

    % commands and environments needed by pandoc snippets
    % extracted from the output of `pandoc -s`
    \providecommand{\tightlist}{%
      \setlength{\itemsep}{0pt}\setlength{\parskip}{0pt}}
    \DefineVerbatimEnvironment{Highlighting}{Verbatim}{commandchars=\\\{\}}
    % Add ',fontsize=\small' for more characters per line
    \newenvironment{Shaded}{}{}
    \newcommand{\KeywordTok}[1]{\textcolor[rgb]{0.00,0.44,0.13}{\textbf{{#1}}}}
    \newcommand{\DataTypeTok}[1]{\textcolor[rgb]{0.56,0.13,0.00}{{#1}}}
    \newcommand{\DecValTok}[1]{\textcolor[rgb]{0.25,0.63,0.44}{{#1}}}
    \newcommand{\BaseNTok}[1]{\textcolor[rgb]{0.25,0.63,0.44}{{#1}}}
    \newcommand{\FloatTok}[1]{\textcolor[rgb]{0.25,0.63,0.44}{{#1}}}
    \newcommand{\CharTok}[1]{\textcolor[rgb]{0.25,0.44,0.63}{{#1}}}
    \newcommand{\StringTok}[1]{\textcolor[rgb]{0.25,0.44,0.63}{{#1}}}
    \newcommand{\CommentTok}[1]{\textcolor[rgb]{0.38,0.63,0.69}{\textit{{#1}}}}
    \newcommand{\OtherTok}[1]{\textcolor[rgb]{0.00,0.44,0.13}{{#1}}}
    \newcommand{\AlertTok}[1]{\textcolor[rgb]{1.00,0.00,0.00}{\textbf{{#1}}}}
    \newcommand{\FunctionTok}[1]{\textcolor[rgb]{0.02,0.16,0.49}{{#1}}}
    \newcommand{\RegionMarkerTok}[1]{{#1}}
    \newcommand{\ErrorTok}[1]{\textcolor[rgb]{1.00,0.00,0.00}{\textbf{{#1}}}}
    \newcommand{\NormalTok}[1]{{#1}}
    
    % Additional commands for more recent versions of Pandoc
    \newcommand{\ConstantTok}[1]{\textcolor[rgb]{0.53,0.00,0.00}{{#1}}}
    \newcommand{\SpecialCharTok}[1]{\textcolor[rgb]{0.25,0.44,0.63}{{#1}}}
    \newcommand{\VerbatimStringTok}[1]{\textcolor[rgb]{0.25,0.44,0.63}{{#1}}}
    \newcommand{\SpecialStringTok}[1]{\textcolor[rgb]{0.73,0.40,0.53}{{#1}}}
    \newcommand{\ImportTok}[1]{{#1}}
    \newcommand{\DocumentationTok}[1]{\textcolor[rgb]{0.73,0.13,0.13}{\textit{{#1}}}}
    \newcommand{\AnnotationTok}[1]{\textcolor[rgb]{0.38,0.63,0.69}{\textbf{\textit{{#1}}}}}
    \newcommand{\CommentVarTok}[1]{\textcolor[rgb]{0.38,0.63,0.69}{\textbf{\textit{{#1}}}}}
    \newcommand{\VariableTok}[1]{\textcolor[rgb]{0.10,0.09,0.49}{{#1}}}
    \newcommand{\ControlFlowTok}[1]{\textcolor[rgb]{0.00,0.44,0.13}{\textbf{{#1}}}}
    \newcommand{\OperatorTok}[1]{\textcolor[rgb]{0.40,0.40,0.40}{{#1}}}
    \newcommand{\BuiltInTok}[1]{{#1}}
    \newcommand{\ExtensionTok}[1]{{#1}}
    \newcommand{\PreprocessorTok}[1]{\textcolor[rgb]{0.74,0.48,0.00}{{#1}}}
    \newcommand{\AttributeTok}[1]{\textcolor[rgb]{0.49,0.56,0.16}{{#1}}}
    \newcommand{\InformationTok}[1]{\textcolor[rgb]{0.38,0.63,0.69}{\textbf{\textit{{#1}}}}}
    \newcommand{\WarningTok}[1]{\textcolor[rgb]{0.38,0.63,0.69}{\textbf{\textit{{#1}}}}}
    
    
    % Define a nice break command that doesn't care if a line doesn't already
    % exist.
    \def\br{\hspace*{\fill} \\* }
    % Math Jax compatibility definitions
    \def\gt{>}
    \def\lt{<}
    \let\Oldtex\TeX
    \let\Oldlatex\LaTeX
    \renewcommand{\TeX}{\textrm{\Oldtex}}
    \renewcommand{\LaTeX}{\textrm{\Oldlatex}}
    % Document parameters
    % Document title
    \title{Dataclasses}
    \author{Kumar Shantanu}
    
    
    
    
    
% Pygments definitions
\makeatletter
\def\PY@reset{\let\PY@it=\relax \let\PY@bf=\relax%
    \let\PY@ul=\relax \let\PY@tc=\relax%
    \let\PY@bc=\relax \let\PY@ff=\relax}
\def\PY@tok#1{\csname PY@tok@#1\endcsname}
\def\PY@toks#1+{\ifx\relax#1\empty\else%
    \PY@tok{#1}\expandafter\PY@toks\fi}
\def\PY@do#1{\PY@bc{\PY@tc{\PY@ul{%
    \PY@it{\PY@bf{\PY@ff{#1}}}}}}}
\def\PY#1#2{\PY@reset\PY@toks#1+\relax+\PY@do{#2}}

\@namedef{PY@tok@w}{\def\PY@tc##1{\textcolor[rgb]{0.73,0.73,0.73}{##1}}}
\@namedef{PY@tok@c}{\let\PY@it=\textit\def\PY@tc##1{\textcolor[rgb]{0.24,0.48,0.48}{##1}}}
\@namedef{PY@tok@cp}{\def\PY@tc##1{\textcolor[rgb]{0.61,0.40,0.00}{##1}}}
\@namedef{PY@tok@k}{\let\PY@bf=\textbf\def\PY@tc##1{\textcolor[rgb]{0.00,0.50,0.00}{##1}}}
\@namedef{PY@tok@kp}{\def\PY@tc##1{\textcolor[rgb]{0.00,0.50,0.00}{##1}}}
\@namedef{PY@tok@kt}{\def\PY@tc##1{\textcolor[rgb]{0.69,0.00,0.25}{##1}}}
\@namedef{PY@tok@o}{\def\PY@tc##1{\textcolor[rgb]{0.40,0.40,0.40}{##1}}}
\@namedef{PY@tok@ow}{\let\PY@bf=\textbf\def\PY@tc##1{\textcolor[rgb]{0.67,0.13,1.00}{##1}}}
\@namedef{PY@tok@nb}{\def\PY@tc##1{\textcolor[rgb]{0.00,0.50,0.00}{##1}}}
\@namedef{PY@tok@nf}{\def\PY@tc##1{\textcolor[rgb]{0.00,0.00,1.00}{##1}}}
\@namedef{PY@tok@nc}{\let\PY@bf=\textbf\def\PY@tc##1{\textcolor[rgb]{0.00,0.00,1.00}{##1}}}
\@namedef{PY@tok@nn}{\let\PY@bf=\textbf\def\PY@tc##1{\textcolor[rgb]{0.00,0.00,1.00}{##1}}}
\@namedef{PY@tok@ne}{\let\PY@bf=\textbf\def\PY@tc##1{\textcolor[rgb]{0.80,0.25,0.22}{##1}}}
\@namedef{PY@tok@nv}{\def\PY@tc##1{\textcolor[rgb]{0.10,0.09,0.49}{##1}}}
\@namedef{PY@tok@no}{\def\PY@tc##1{\textcolor[rgb]{0.53,0.00,0.00}{##1}}}
\@namedef{PY@tok@nl}{\def\PY@tc##1{\textcolor[rgb]{0.46,0.46,0.00}{##1}}}
\@namedef{PY@tok@ni}{\let\PY@bf=\textbf\def\PY@tc##1{\textcolor[rgb]{0.44,0.44,0.44}{##1}}}
\@namedef{PY@tok@na}{\def\PY@tc##1{\textcolor[rgb]{0.41,0.47,0.13}{##1}}}
\@namedef{PY@tok@nt}{\let\PY@bf=\textbf\def\PY@tc##1{\textcolor[rgb]{0.00,0.50,0.00}{##1}}}
\@namedef{PY@tok@nd}{\def\PY@tc##1{\textcolor[rgb]{0.67,0.13,1.00}{##1}}}
\@namedef{PY@tok@s}{\def\PY@tc##1{\textcolor[rgb]{0.73,0.13,0.13}{##1}}}
\@namedef{PY@tok@sd}{\let\PY@it=\textit\def\PY@tc##1{\textcolor[rgb]{0.73,0.13,0.13}{##1}}}
\@namedef{PY@tok@si}{\let\PY@bf=\textbf\def\PY@tc##1{\textcolor[rgb]{0.64,0.35,0.47}{##1}}}
\@namedef{PY@tok@se}{\let\PY@bf=\textbf\def\PY@tc##1{\textcolor[rgb]{0.67,0.36,0.12}{##1}}}
\@namedef{PY@tok@sr}{\def\PY@tc##1{\textcolor[rgb]{0.64,0.35,0.47}{##1}}}
\@namedef{PY@tok@ss}{\def\PY@tc##1{\textcolor[rgb]{0.10,0.09,0.49}{##1}}}
\@namedef{PY@tok@sx}{\def\PY@tc##1{\textcolor[rgb]{0.00,0.50,0.00}{##1}}}
\@namedef{PY@tok@m}{\def\PY@tc##1{\textcolor[rgb]{0.40,0.40,0.40}{##1}}}
\@namedef{PY@tok@gh}{\let\PY@bf=\textbf\def\PY@tc##1{\textcolor[rgb]{0.00,0.00,0.50}{##1}}}
\@namedef{PY@tok@gu}{\let\PY@bf=\textbf\def\PY@tc##1{\textcolor[rgb]{0.50,0.00,0.50}{##1}}}
\@namedef{PY@tok@gd}{\def\PY@tc##1{\textcolor[rgb]{0.63,0.00,0.00}{##1}}}
\@namedef{PY@tok@gi}{\def\PY@tc##1{\textcolor[rgb]{0.00,0.52,0.00}{##1}}}
\@namedef{PY@tok@gr}{\def\PY@tc##1{\textcolor[rgb]{0.89,0.00,0.00}{##1}}}
\@namedef{PY@tok@ge}{\let\PY@it=\textit}
\@namedef{PY@tok@gs}{\let\PY@bf=\textbf}
\@namedef{PY@tok@gp}{\let\PY@bf=\textbf\def\PY@tc##1{\textcolor[rgb]{0.00,0.00,0.50}{##1}}}
\@namedef{PY@tok@go}{\def\PY@tc##1{\textcolor[rgb]{0.44,0.44,0.44}{##1}}}
\@namedef{PY@tok@gt}{\def\PY@tc##1{\textcolor[rgb]{0.00,0.27,0.87}{##1}}}
\@namedef{PY@tok@err}{\def\PY@bc##1{{\setlength{\fboxsep}{\string -\fboxrule}\fcolorbox[rgb]{1.00,0.00,0.00}{1,1,1}{\strut ##1}}}}
\@namedef{PY@tok@kc}{\let\PY@bf=\textbf\def\PY@tc##1{\textcolor[rgb]{0.00,0.50,0.00}{##1}}}
\@namedef{PY@tok@kd}{\let\PY@bf=\textbf\def\PY@tc##1{\textcolor[rgb]{0.00,0.50,0.00}{##1}}}
\@namedef{PY@tok@kn}{\let\PY@bf=\textbf\def\PY@tc##1{\textcolor[rgb]{0.00,0.50,0.00}{##1}}}
\@namedef{PY@tok@kr}{\let\PY@bf=\textbf\def\PY@tc##1{\textcolor[rgb]{0.00,0.50,0.00}{##1}}}
\@namedef{PY@tok@bp}{\def\PY@tc##1{\textcolor[rgb]{0.00,0.50,0.00}{##1}}}
\@namedef{PY@tok@fm}{\def\PY@tc##1{\textcolor[rgb]{0.00,0.00,1.00}{##1}}}
\@namedef{PY@tok@vc}{\def\PY@tc##1{\textcolor[rgb]{0.10,0.09,0.49}{##1}}}
\@namedef{PY@tok@vg}{\def\PY@tc##1{\textcolor[rgb]{0.10,0.09,0.49}{##1}}}
\@namedef{PY@tok@vi}{\def\PY@tc##1{\textcolor[rgb]{0.10,0.09,0.49}{##1}}}
\@namedef{PY@tok@vm}{\def\PY@tc##1{\textcolor[rgb]{0.10,0.09,0.49}{##1}}}
\@namedef{PY@tok@sa}{\def\PY@tc##1{\textcolor[rgb]{0.73,0.13,0.13}{##1}}}
\@namedef{PY@tok@sb}{\def\PY@tc##1{\textcolor[rgb]{0.73,0.13,0.13}{##1}}}
\@namedef{PY@tok@sc}{\def\PY@tc##1{\textcolor[rgb]{0.73,0.13,0.13}{##1}}}
\@namedef{PY@tok@dl}{\def\PY@tc##1{\textcolor[rgb]{0.73,0.13,0.13}{##1}}}
\@namedef{PY@tok@s2}{\def\PY@tc##1{\textcolor[rgb]{0.73,0.13,0.13}{##1}}}
\@namedef{PY@tok@sh}{\def\PY@tc##1{\textcolor[rgb]{0.73,0.13,0.13}{##1}}}
\@namedef{PY@tok@s1}{\def\PY@tc##1{\textcolor[rgb]{0.73,0.13,0.13}{##1}}}
\@namedef{PY@tok@mb}{\def\PY@tc##1{\textcolor[rgb]{0.40,0.40,0.40}{##1}}}
\@namedef{PY@tok@mf}{\def\PY@tc##1{\textcolor[rgb]{0.40,0.40,0.40}{##1}}}
\@namedef{PY@tok@mh}{\def\PY@tc##1{\textcolor[rgb]{0.40,0.40,0.40}{##1}}}
\@namedef{PY@tok@mi}{\def\PY@tc##1{\textcolor[rgb]{0.40,0.40,0.40}{##1}}}
\@namedef{PY@tok@il}{\def\PY@tc##1{\textcolor[rgb]{0.40,0.40,0.40}{##1}}}
\@namedef{PY@tok@mo}{\def\PY@tc##1{\textcolor[rgb]{0.40,0.40,0.40}{##1}}}
\@namedef{PY@tok@ch}{\let\PY@it=\textit\def\PY@tc##1{\textcolor[rgb]{0.24,0.48,0.48}{##1}}}
\@namedef{PY@tok@cm}{\let\PY@it=\textit\def\PY@tc##1{\textcolor[rgb]{0.24,0.48,0.48}{##1}}}
\@namedef{PY@tok@cpf}{\let\PY@it=\textit\def\PY@tc##1{\textcolor[rgb]{0.24,0.48,0.48}{##1}}}
\@namedef{PY@tok@c1}{\let\PY@it=\textit\def\PY@tc##1{\textcolor[rgb]{0.24,0.48,0.48}{##1}}}
\@namedef{PY@tok@cs}{\let\PY@it=\textit\def\PY@tc##1{\textcolor[rgb]{0.24,0.48,0.48}{##1}}}

\def\PYZbs{\char`\\}
\def\PYZus{\char`\_}
\def\PYZob{\char`\{}
\def\PYZcb{\char`\}}
\def\PYZca{\char`\^}
\def\PYZam{\char`\&}
\def\PYZlt{\char`\<}
\def\PYZgt{\char`\>}
\def\PYZsh{\char`\#}
\def\PYZpc{\char`\%}
\def\PYZdl{\char`\$}
\def\PYZhy{\char`\-}
\def\PYZsq{\char`\'}
\def\PYZdq{\char`\"}
\def\PYZti{\char`\~}
% for compatibility with earlier versions
\def\PYZat{@}
\def\PYZlb{[}
\def\PYZrb{]}
\makeatother


    % For linebreaks inside Verbatim environment from package fancyvrb. 
    \makeatletter
        \newbox\Wrappedcontinuationbox 
        \newbox\Wrappedvisiblespacebox 
        \newcommand*\Wrappedvisiblespace {\textcolor{red}{\textvisiblespace}} 
        \newcommand*\Wrappedcontinuationsymbol {\textcolor{red}{\llap{\tiny$\m@th\hookrightarrow$}}} 
        \newcommand*\Wrappedcontinuationindent {3ex } 
        \newcommand*\Wrappedafterbreak {\kern\Wrappedcontinuationindent\copy\Wrappedcontinuationbox} 
        % Take advantage of the already applied Pygments mark-up to insert 
        % potential linebreaks for TeX processing. 
        %        {, <, #, %, $, ' and ": go to next line. 
        %        _, }, ^, &, >, - and ~: stay at end of broken line. 
        % Use of \textquotesingle for straight quote. 
        \newcommand*\Wrappedbreaksatspecials {% 
            \def\PYGZus{\discretionary{\char`\_}{\Wrappedafterbreak}{\char`\_}}% 
            \def\PYGZob{\discretionary{}{\Wrappedafterbreak\char`\{}{\char`\{}}% 
            \def\PYGZcb{\discretionary{\char`\}}{\Wrappedafterbreak}{\char`\}}}% 
            \def\PYGZca{\discretionary{\char`\^}{\Wrappedafterbreak}{\char`\^}}% 
            \def\PYGZam{\discretionary{\char`\&}{\Wrappedafterbreak}{\char`\&}}% 
            \def\PYGZlt{\discretionary{}{\Wrappedafterbreak\char`\<}{\char`\<}}% 
            \def\PYGZgt{\discretionary{\char`\>}{\Wrappedafterbreak}{\char`\>}}% 
            \def\PYGZsh{\discretionary{}{\Wrappedafterbreak\char`\#}{\char`\#}}% 
            \def\PYGZpc{\discretionary{}{\Wrappedafterbreak\char`\%}{\char`\%}}% 
            \def\PYGZdl{\discretionary{}{\Wrappedafterbreak\char`\$}{\char`\$}}% 
            \def\PYGZhy{\discretionary{\char`\-}{\Wrappedafterbreak}{\char`\-}}% 
            \def\PYGZsq{\discretionary{}{\Wrappedafterbreak\textquotesingle}{\textquotesingle}}% 
            \def\PYGZdq{\discretionary{}{\Wrappedafterbreak\char`\"}{\char`\"}}% 
            \def\PYGZti{\discretionary{\char`\~}{\Wrappedafterbreak}{\char`\~}}% 
        } 
        % Some characters . , ; ? ! / are not pygmentized. 
        % This macro makes them "active" and they will insert potential linebreaks 
        \newcommand*\Wrappedbreaksatpunct {% 
            \lccode`\~`\.\lowercase{\def~}{\discretionary{\hbox{\char`\.}}{\Wrappedafterbreak}{\hbox{\char`\.}}}% 
            \lccode`\~`\,\lowercase{\def~}{\discretionary{\hbox{\char`\,}}{\Wrappedafterbreak}{\hbox{\char`\,}}}% 
            \lccode`\~`\;\lowercase{\def~}{\discretionary{\hbox{\char`\;}}{\Wrappedafterbreak}{\hbox{\char`\;}}}% 
            \lccode`\~`\:\lowercase{\def~}{\discretionary{\hbox{\char`\:}}{\Wrappedafterbreak}{\hbox{\char`\:}}}% 
            \lccode`\~`\?\lowercase{\def~}{\discretionary{\hbox{\char`\?}}{\Wrappedafterbreak}{\hbox{\char`\?}}}% 
            \lccode`\~`\!\lowercase{\def~}{\discretionary{\hbox{\char`\!}}{\Wrappedafterbreak}{\hbox{\char`\!}}}% 
            \lccode`\~`\/\lowercase{\def~}{\discretionary{\hbox{\char`\/}}{\Wrappedafterbreak}{\hbox{\char`\/}}}% 
            \catcode`\.\active
            \catcode`\,\active 
            \catcode`\;\active
            \catcode`\:\active
            \catcode`\?\active
            \catcode`\!\active
            \catcode`\/\active 
            \lccode`\~`\~ 	
        }
    \makeatother

    \let\OriginalVerbatim=\Verbatim
    \makeatletter
    \renewcommand{\Verbatim}[1][1]{%
        %\parskip\z@skip
        \sbox\Wrappedcontinuationbox {\Wrappedcontinuationsymbol}%
        \sbox\Wrappedvisiblespacebox {\FV@SetupFont\Wrappedvisiblespace}%
        \def\FancyVerbFormatLine ##1{\hsize\linewidth
            \vtop{\raggedright\hyphenpenalty\z@\exhyphenpenalty\z@
                \doublehyphendemerits\z@\finalhyphendemerits\z@
                \strut ##1\strut}%
        }%
        % If the linebreak is at a space, the latter will be displayed as visible
        % space at end of first line, and a continuation symbol starts next line.
        % Stretch/shrink are however usually zero for typewriter font.
        \def\FV@Space {%
            \nobreak\hskip\z@ plus\fontdimen3\font minus\fontdimen4\font
            \discretionary{\copy\Wrappedvisiblespacebox}{\Wrappedafterbreak}
            {\kern\fontdimen2\font}%
        }%
        
        % Allow breaks at special characters using \PYG... macros.
        \Wrappedbreaksatspecials
        % Breaks at punctuation characters . , ; ? ! and / need catcode=\active 	
        \OriginalVerbatim[#1,codes*=\Wrappedbreaksatpunct]%
    }
    \makeatother

    % Exact colors from NB
    \definecolor{incolor}{HTML}{303F9F}
    \definecolor{outcolor}{HTML}{D84315}
    \definecolor{cellborder}{HTML}{CFCFCF}
    \definecolor{cellbackground}{HTML}{F7F7F7}
    
    % prompt
    \makeatletter
    \newcommand{\boxspacing}{\kern\kvtcb@left@rule\kern\kvtcb@boxsep}
    \makeatother
    \newcommand{\prompt}[4]{
        {\ttfamily\llap{{\color{#2}[#3]:\hspace{3pt}#4}}\vspace{-\baselineskip}}
    }
    

    
    % Prevent overflowing lines due to hard-to-break entities
    \sloppy 
    % Setup hyperref package
    \hypersetup{
      breaklinks=true,  % so long urls are correctly broken across lines
      colorlinks=true,
      urlcolor=urlcolor,
      linkcolor=linkcolor,
      citecolor=citecolor,
      }
    % Slightly bigger margins than the latex defaults
    
    \geometry{verbose,tmargin=1in,bmargin=1in,lmargin=1in,rmargin=1in}
    
    

\begin{document}
    
    \maketitle
    
    

    
    \hypertarget{dataclasses-in-python}{%
\section{Dataclasses in Python}\label{dataclasses-in-python}}

This tutorial assumes that you already know the basics of
\href{https://en.wikipedia.org/wiki/Object-oriented_programming}{object-oriented
programming} in general and have a working knowledge of Python. If you
are not familiar with object-oriented programming, you should read the
tutorial on Python classes first.

Useful links:
https://realpython.com/python3-object-oriented-programming/

\hypertarget{what-are-dataclasses}{%
\subsection{What are dataclasses?}\label{what-are-dataclasses}}

Dataclasses are a way to create classes that are lightweight and have a
lot of the functionality of a class without the overhead of a class.
They are useful for creating classes that are used to store data. They
are also useful for creating classes that are used to store data that is
used to create other classes. The dataclasses are available through
built-in dataclasses module in Python 3.7+.

    \begin{tcolorbox}[breakable, size=fbox, boxrule=1pt, pad at break*=1mm,colback=cellbackground, colframe=cellborder]
\prompt{In}{incolor}{2}{\boxspacing}
\begin{Verbatim}[commandchars=\\\{\}]
\PY{k+kn}{from} \PY{n+nn}{dataclasses} \PY{k+kn}{import} \PY{n}{dataclass}
\end{Verbatim}
\end{tcolorbox}

    Let us create a dataclass that represents a bond. This can be achieved
as follows:

    \begin{tcolorbox}[breakable, size=fbox, boxrule=1pt, pad at break*=1mm,colback=cellbackground, colframe=cellborder]
\prompt{In}{incolor}{4}{\boxspacing}
\begin{Verbatim}[commandchars=\\\{\}]
\PY{n+nd}{@dataclass}
\PY{k}{class} \PY{n+nc}{Bond}\PY{p}{:}
  \PY{l+s+sd}{\PYZdq{}\PYZdq{}\PYZdq{}}
\PY{l+s+sd}{  The dataclass that represents a bond.}
\PY{l+s+sd}{  }
\PY{l+s+sd}{  \PYZsh{}\PYZsh{} Parameters:}
\PY{l+s+sd}{  rate: The rate of interest on the bond.}
\PY{l+s+sd}{  duration: The duration of the bond.}
\PY{l+s+sd}{  face\PYZus{}value: The face value of the bond. }
\PY{l+s+sd}{  }
\PY{l+s+sd}{  \PYZsh{}\PYZsh{} Properties:}
\PY{l+s+sd}{  price: The present value of the future cash flows. }
\PY{l+s+sd}{  \PYZdq{}\PYZdq{}\PYZdq{}}  
  \PY{n}{rate}\PY{p}{:} \PY{n+nb}{float}
  \PY{n}{duration}\PY{p}{:} \PY{n+nb}{float}
  \PY{n}{face\PYZus{}value}\PY{p}{:} \PY{n+nb}{float}
  
  \PY{n+nd}{@property}
  \PY{k}{def} \PY{n+nf}{price}\PY{p}{(}\PY{n+nb+bp}{self}\PY{p}{)} \PY{o}{\PYZhy{}}\PY{o}{\PYZgt{}} \PY{n+nb}{float}\PY{p}{:}    
    \PY{k}{return} \PY{n+nb+bp}{self}\PY{o}{.}\PY{n}{face\PYZus{}value} \PY{o}{/} \PY{p}{(}\PY{l+m+mi}{1} \PY{o}{+} \PY{n+nb+bp}{self}\PY{o}{.}\PY{n}{rate}\PY{p}{)} \PY{o}{*}\PY{o}{*} \PY{n+nb+bp}{self}\PY{o}{.}\PY{n}{duration}
\end{Verbatim}
\end{tcolorbox}

    The property decorator is used to store a computed attribute within a
dataclass. The same code logic as above but without the dataclass
decorator.

    \begin{tcolorbox}[breakable, size=fbox, boxrule=1pt, pad at break*=1mm,colback=cellbackground, colframe=cellborder]
\prompt{In}{incolor}{ }{\boxspacing}
\begin{Verbatim}[commandchars=\\\{\}]
\PY{k}{class} \PY{n+nc}{Bond}\PY{p}{:}
  \PY{k}{def} \PY{n+nf+fm}{\PYZus{}\PYZus{}init\PYZus{}\PYZus{}}\PY{p}{(}\PY{n+nb+bp}{self}\PY{p}{)} \PY{o}{\PYZhy{}}\PY{o}{\PYZgt{}} \PY{k+kc}{None}\PY{p}{:}
    \PY{n+nb+bp}{self}\PY{o}{.}\PY{n}{rate} \PY{o}{=} \PY{l+m+mf}{0.0}
    \PY{n+nb+bp}{self}\PY{o}{.}\PY{n}{duration} \PY{o}{=} \PY{l+m+mf}{0.0}
    \PY{n+nb+bp}{self}\PY{o}{.}\PY{n}{face\PYZus{}value} \PY{o}{=} \PY{l+m+mf}{0.0}
  
  \PY{k}{def} \PY{n+nf}{CalculatePresentValue}\PY{p}{(}\PY{n+nb+bp}{self}\PY{p}{)} \PY{o}{\PYZhy{}}\PY{o}{\PYZgt{}} \PY{k+kc}{None}\PY{p}{:}
    \PY{n+nb+bp}{self}\PY{o}{.}\PY{n}{price} \PY{o}{=} \PY{n+nb+bp}{self}\PY{o}{.}\PY{n}{face\PYZus{}value} \PY{o}{/} \PY{p}{(}\PY{l+m+mi}{1} \PY{o}{+} \PY{n+nb+bp}{self}\PY{o}{.}\PY{n}{rate}\PY{p}{)} \PY{o}{*}\PY{o}{*} \PY{n+nb+bp}{self}\PY{o}{.}\PY{n}{duration}
  
\end{Verbatim}
\end{tcolorbox}

    \hypertarget{why-should-you-care-about-dataclasses-when-you-have-simpler-tabular-datastructures-like-pandas-dataframes}{%
\subsection{Why should you care about dataclasses when you have simpler
tabular datastructures like Pandas
dataframes?}\label{why-should-you-care-about-dataclasses-when-you-have-simpler-tabular-datastructures-like-pandas-dataframes}}

One of the most powerful features with the Python dataclasses are that
they support many features of object oriented programming. This would
become clearer with the example of mortgages below. Mortgages are also a
subclass of bonds, therefore we will simply inherit Bonds into a new
dataclass called mortgages.

    \begin{tcolorbox}[breakable, size=fbox, boxrule=1pt, pad at break*=1mm,colback=cellbackground, colframe=cellborder]
\prompt{In}{incolor}{110}{\boxspacing}
\begin{Verbatim}[commandchars=\\\{\}]
\PY{n+nd}{@dataclass}
\PY{k}{class} \PY{n+nc}{Mortgage} \PY{p}{(}\PY{n}{Bond}\PY{p}{)}\PY{p}{:}
  \PY{k}{pass}
\end{Verbatim}
\end{tcolorbox}

    Now you can probably see that we have a structure. We have implemented
programitacally that a mortgage is actually a bond and it can also have
additional features. Mortgages will also have attributes that are
specific to mortgages. We will add the down payment and the monthly
payment to the mortgage class. Let us add those features and define the
mortgage class again.

    \begin{tcolorbox}[breakable, size=fbox, boxrule=1pt, pad at break*=1mm,colback=cellbackground, colframe=cellborder]
\prompt{In}{incolor}{123}{\boxspacing}
\begin{Verbatim}[commandchars=\\\{\}]
\PY{n+nd}{@dataclass}
\PY{k}{class} \PY{n+nc}{Mortgage}\PY{p}{(}\PY{n}{Bond}\PY{p}{)}\PY{p}{:}  
  \PY{n}{down\PYZus{}payment}\PY{p}{:} \PY{n+nb}{float}
  \PY{n+nd}{@property}
  \PY{k}{def} \PY{n+nf}{monthly\PYZus{}payment}\PY{p}{(}\PY{n+nb+bp}{self}\PY{p}{)} \PY{o}{\PYZhy{}}\PY{o}{\PYZgt{}} \PY{n+nb}{float}\PY{p}{:}
    \PY{n}{monthly\PYZus{}rate} \PY{o}{=} \PY{n+nb+bp}{self}\PY{o}{.}\PY{n}{rate} \PY{o}{/} \PY{l+m+mi}{12}
    \PY{k}{return} \PY{p}{(}\PY{n+nb+bp}{self}\PY{o}{.}\PY{n}{face\PYZus{}value} \PY{o}{\PYZhy{}} \PY{n+nb+bp}{self}\PY{o}{.}\PY{n}{down\PYZus{}payment}\PY{p}{)} \PY{o}{*} \PY{n}{monthly\PYZus{}rate} \PY{o}{/} \PY{p}{(}\PY{l+m+mi}{1} \PY{o}{\PYZhy{}} \PY{p}{(}\PY{l+m+mi}{1} \PY{o}{+} \PY{n}{monthly\PYZus{}rate}\PY{p}{)} \PY{o}{*}\PY{o}{*} \PY{p}{(}\PY{o}{\PYZhy{}}\PY{n+nb+bp}{self}\PY{o}{.}\PY{n}{duration} \PY{o}{*} \PY{l+m+mi}{12}\PY{p}{)}\PY{p}{)} 
  \PY{c+c1}{\PYZsh{} We also need to override the price property. The price of the mortgage is the present value of the monthly payments plus the down payment. This is equal to the face value or the amount borrowed.}
  \PY{n+nd}{@property}
  \PY{k}{def} \PY{n+nf}{price}\PY{p}{(}\PY{n+nb+bp}{self}\PY{p}{)}\PY{p}{:}
    \PY{k}{return} \PY{n+nb+bp}{self}\PY{o}{.}\PY{n}{face\PYZus{}value}
\end{Verbatim}
\end{tcolorbox}

    Now let us create a mortgage object. Note that we need to supply 4
parameters: the rate, duration, face value and down payment. Three of
these parameters are inherited from the Bond class. The down payment is
a new parameter that is specific to the Mortgage class. Mortgage class
also has an additional property attribute called monthly payment.

    \begin{tcolorbox}[breakable, size=fbox, boxrule=1pt, pad at break*=1mm,colback=cellbackground, colframe=cellborder]
\prompt{In}{incolor}{124}{\boxspacing}
\begin{Verbatim}[commandchars=\\\{\}]
\PY{n}{fixed\PYZus{}mortgage} \PY{o}{=} \PY{n}{Mortgage}\PY{p}{(}\PY{n}{rate}\PY{o}{=}\PY{l+m+mf}{0.05}\PY{p}{,} \PY{n}{duration}\PY{o}{=}\PY{l+m+mi}{30}\PY{p}{,} \PY{n}{face\PYZus{}value}\PY{o}{=}\PY{l+m+mi}{100000}\PY{p}{,} \PY{n}{down\PYZus{}payment}\PY{o}{=}\PY{l+m+mi}{20000}\PY{p}{)}
\PY{n}{fixed\PYZus{}mortgage}\PY{o}{.}\PY{n}{monthly\PYZus{}payment}
\end{Verbatim}
\end{tcolorbox}

            \begin{tcolorbox}[breakable, size=fbox, boxrule=.5pt, pad at break*=1mm, opacityfill=0]
\prompt{Out}{outcolor}{124}{\boxspacing}
\begin{Verbatim}[commandchars=\\\{\}]
429.4572984097118
\end{Verbatim}
\end{tcolorbox}
        
    But mortgages can either be fixed rate or floating rate. We will now
inherit from mortgage and create a floating rate mortgage. Now you could
be probably getting a feel of how powerful dataclasses could be.

    \begin{tcolorbox}[breakable, size=fbox, boxrule=1pt, pad at break*=1mm,colback=cellbackground, colframe=cellborder]
\prompt{In}{incolor}{114}{\boxspacing}
\begin{Verbatim}[commandchars=\\\{\}]
\PY{n+nd}{@dataclass}
\PY{k}{class} \PY{n+nc}{FloatingRateMortgage}\PY{p}{(}\PY{n}{Mortgage}\PY{p}{)}\PY{p}{:}  
  \PY{n+nd}{@property}
  \PY{k}{def} \PY{n+nf}{monthly\PYZus{}payment}\PY{p}{(}\PY{n+nb+bp}{self}\PY{p}{)} \PY{o}{\PYZhy{}}\PY{o}{\PYZgt{}} \PY{n+nb}{float}\PY{p}{:}
      \PY{k}{if}\PY{p}{(}\PY{n+nb}{len}\PY{p}{(}\PY{n+nb+bp}{self}\PY{o}{.}\PY{n}{rate}\PY{p}{)} \PY{o}{!=} \PY{n+nb+bp}{self}\PY{o}{.}\PY{n}{duration} \PY{o}{*} \PY{l+m+mi}{12}\PY{p}{)}\PY{p}{:}
          \PY{k}{raise} \PY{n+ne}{ValueError}\PY{p}{(}\PY{l+s+s2}{\PYZdq{}}\PY{l+s+s2}{The number of monthly rates is not equal to the number of months in the mortgage duration.}\PY{l+s+s2}{\PYZdq{}}\PY{p}{)}
      \PY{c+c1}{\PYZsh{} Now we will have a list of monthly payments. We will have to calculate the monthly payment for each month based on the monthly rate.}
      \PY{k}{return} \PY{p}{[}\PY{p}{(}\PY{n+nb+bp}{self}\PY{o}{.}\PY{n}{face\PYZus{}value} \PY{o}{\PYZhy{}} \PY{n+nb+bp}{self}\PY{o}{.}\PY{n}{down\PYZus{}payment}\PY{p}{)} \PY{o}{*} \PY{p}{(}\PY{n}{rate}\PY{o}{/}\PY{l+m+mi}{12}\PY{p}{)} \PY{o}{/} \PY{p}{(}\PY{l+m+mi}{1} \PY{o}{\PYZhy{}} \PY{p}{(}\PY{l+m+mi}{1} \PY{o}{+} \PY{p}{(}\PY{n}{rate}\PY{o}{/}\PY{l+m+mi}{12}\PY{p}{)}\PY{p}{)} \PY{o}{*}\PY{o}{*} \PY{p}{(}\PY{o}{\PYZhy{}}\PY{n+nb+bp}{self}\PY{o}{.}\PY{n}{duration} \PY{o}{*} \PY{l+m+mi}{12}\PY{p}{)}\PY{p}{)} \PY{k}{for} \PY{n}{rate} \PY{o+ow}{in} \PY{n+nb+bp}{self}\PY{o}{.}\PY{n}{rate}\PY{p}{]}   
\end{Verbatim}
\end{tcolorbox}

    We will now create a floating rate mortgage object. We will assume that
the monthly annualised rate of interest is
\href{https://en.wikipedia.org/wiki/Continuous_uniform_distribution}{uniformly
distributed} between 4\% and 12\%. Note that the rate of interest is
annualised and this annualised rate changes every month.

    \begin{tcolorbox}[breakable, size=fbox, boxrule=1pt, pad at break*=1mm,colback=cellbackground, colframe=cellborder]
\prompt{In}{incolor}{115}{\boxspacing}
\begin{Verbatim}[commandchars=\\\{\}]
\PY{k+kn}{from} \PY{n+nn}{numpy}\PY{n+nn}{.}\PY{n+nn}{random} \PY{k+kn}{import} \PY{n}{uniform}
\PY{n}{floating\PYZus{}mortgage} \PY{o}{=} \PY{n}{FloatingRateMortgage}\PY{p}{(}\PY{n}{duration}\PY{o}{=}\PY{l+m+mi}{30}\PY{p}{,} \PY{n}{face\PYZus{}value}\PY{o}{=}\PY{l+m+mi}{100000}\PY{p}{,} \PY{n}{down\PYZus{}payment}\PY{o}{=}\PY{l+m+mi}{20000}\PY{p}{,} \PY{n}{rate}\PY{o}{=}\PY{n}{uniform}\PY{p}{(}\PY{l+m+mf}{0.04}\PY{p}{,} \PY{l+m+mf}{0.12}\PY{p}{,} \PY{l+m+mi}{30}\PY{o}{*}\PY{l+m+mi}{12}\PY{p}{)}\PY{p}{)}
\PY{n}{floating\PYZus{}mortgage}\PY{o}{.}\PY{n}{monthly\PYZus{}payment}\PY{p}{[}\PY{l+m+mi}{0}\PY{p}{:}\PY{l+m+mi}{12}\PY{p}{]} \PY{c+c1}{\PYZsh{} We will print the first year of monthly payments for sake of brevity.}
\end{Verbatim}
\end{tcolorbox}

            \begin{tcolorbox}[breakable, size=fbox, boxrule=.5pt, pad at break*=1mm, opacityfill=0]
\prompt{Out}{outcolor}{115}{\boxspacing}
\begin{Verbatim}[commandchars=\\\{\}]
[530.9941898888909,
 775.2555352740925,
 439.5470626720084,
 761.7033410945966,
 443.19156205169156,
 610.4539485787051,
 475.38375837483676,
 595.064992373187,
 507.8541106689975,
 471.5091034239094,
 547.3519518483176,
 390.21903287103135]
\end{Verbatim}
\end{tcolorbox}
        
    In the above example, we assumed that the rate of interest is uniformly
distributed between 4\% and 12\%. This is quite unrealistic, but that is
not the point. The point is to show, how powerful and convenient
datatclasses can be. We used dataclasses to create a class (eg:
Mortgage) that inherits from another class (eg: Bond) and add additional
attributes and methods to the inherited class. By the way, you can also
use your own interest rates or source the list of interest rates from
any API such as FRED. You can also extend this code to create a useful
application that would simulate different interest rate environments and
show you how the mortgage payments would change or maybe something else
(imagination is the limit!). That is the beauty of Python! You can do
powerful things with simple and readable tools.

    Now let us look at other features that dataclasses provide. Suppose you
are a small bank and you have given out 5 housing loans, with amounts of
\$10000, \$50000, \$60000, \$90000 and \$100000 respectively. All of
these loans have a fixed interest rate of 5\% and and a fixed duration
of 30 years. The down payment for each loan is 20\% of the loan amount.

    \begin{tcolorbox}[breakable, size=fbox, boxrule=1pt, pad at break*=1mm,colback=cellbackground, colframe=cellborder]
\prompt{In}{incolor}{116}{\boxspacing}
\begin{Verbatim}[commandchars=\\\{\}]
\PY{n}{loans} \PY{o}{=} \PY{p}{[}\PY{l+m+mi}{10000}\PY{p}{,} \PY{l+m+mi}{50000}\PY{p}{,} \PY{l+m+mi}{60000}\PY{p}{,} \PY{l+m+mi}{90000}\PY{p}{,} \PY{l+m+mi}{100000}\PY{p}{]}
\PY{n}{mortgage\PYZus{}portfolio} \PY{o}{=} \PY{p}{[}\PY{n}{Mortgage}\PY{p}{(}\PY{n}{rate}\PY{o}{=}\PY{l+m+mf}{0.05}\PY{p}{,} \PY{n}{duration}\PY{o}{=}\PY{l+m+mi}{30}\PY{p}{,} \PY{n}{face\PYZus{}value}\PY{o}{=}\PY{n}{face\PYZus{}value}\PY{p}{,} \PY{n}{down\PYZus{}payment}\PY{o}{=}\PY{l+m+mf}{0.2} \PY{o}{*} \PY{n}{face\PYZus{}value}\PY{p}{)} \PY{k}{for} \PY{n}{face\PYZus{}value} \PY{o+ow}{in} \PY{n}{loans}\PY{p}{]}
\PY{n}{mortgage\PYZus{}portfolio}
\end{Verbatim}
\end{tcolorbox}

            \begin{tcolorbox}[breakable, size=fbox, boxrule=.5pt, pad at break*=1mm, opacityfill=0]
\prompt{Out}{outcolor}{116}{\boxspacing}
\begin{Verbatim}[commandchars=\\\{\}]
[Mortgage(rate=0.05, duration=30, face\_value=10000, down\_payment=2000.0),
 Mortgage(rate=0.05, duration=30, face\_value=50000, down\_payment=10000.0),
 Mortgage(rate=0.05, duration=30, face\_value=60000, down\_payment=12000.0),
 Mortgage(rate=0.05, duration=30, face\_value=90000, down\_payment=18000.0),
 Mortgage(rate=0.05, duration=30, face\_value=100000, down\_payment=20000.0)]
\end{Verbatim}
\end{tcolorbox}
        
    \hypertarget{data-wrangling-with-dataclasses}{%
\subsection{Data Wrangling with
dataclasses}\label{data-wrangling-with-dataclasses}}

We have created a list of mortgage objects. Suppose we want to filter
out the mortgages that have a face value greater than \$50,000.
Filtering the list of dataclasses is same as filtering any other list.

    \begin{tcolorbox}[breakable, size=fbox, boxrule=1pt, pad at break*=1mm,colback=cellbackground, colframe=cellborder]
\prompt{In}{incolor}{117}{\boxspacing}
\begin{Verbatim}[commandchars=\\\{\}]
\PY{n}{filtered\PYZus{}portfolio} \PY{o}{=} \PY{n+nb}{list}\PY{p}{(}\PY{n+nb}{filter}\PY{p}{(}\PY{k}{lambda} \PY{n}{loan}\PY{p}{:} \PY{n}{loan}\PY{o}{.}\PY{n}{face\PYZus{}value} \PY{o}{\PYZgt{}} \PY{l+m+mi}{50000}\PY{p}{,} \PY{n}{mortgage\PYZus{}portfolio}\PY{p}{)}\PY{p}{)}
\PY{n}{filtered\PYZus{}portfolio}
\end{Verbatim}
\end{tcolorbox}

            \begin{tcolorbox}[breakable, size=fbox, boxrule=.5pt, pad at break*=1mm, opacityfill=0]
\prompt{Out}{outcolor}{117}{\boxspacing}
\begin{Verbatim}[commandchars=\\\{\}]
[Mortgage(rate=0.05, duration=30, face\_value=60000, down\_payment=12000.0),
 Mortgage(rate=0.05, duration=30, face\_value=90000, down\_payment=18000.0),
 Mortgage(rate=0.05, duration=30, face\_value=100000, down\_payment=20000.0)]
\end{Verbatim}
\end{tcolorbox}
        
    Or maybe you want to filter only those portfolios that have a monthly
payment greater than \$100

    \begin{tcolorbox}[breakable, size=fbox, boxrule=1pt, pad at break*=1mm,colback=cellbackground, colframe=cellborder]
\prompt{In}{incolor}{118}{\boxspacing}
\begin{Verbatim}[commandchars=\\\{\}]
\PY{n}{filtered\PYZus{}portfolio} \PY{o}{=} \PY{n+nb}{list}\PY{p}{(}\PY{n+nb}{filter}\PY{p}{(}\PY{k}{lambda} \PY{n}{loan}\PY{p}{:} \PY{n}{loan}\PY{o}{.}\PY{n}{monthly\PYZus{}payment} \PY{o}{\PYZgt{}} \PY{l+m+mi}{100}\PY{p}{,} \PY{n}{mortgage\PYZus{}portfolio}\PY{p}{)}\PY{p}{)}
\PY{n}{filtered\PYZus{}portfolio}
\end{Verbatim}
\end{tcolorbox}

            \begin{tcolorbox}[breakable, size=fbox, boxrule=.5pt, pad at break*=1mm, opacityfill=0]
\prompt{Out}{outcolor}{118}{\boxspacing}
\begin{Verbatim}[commandchars=\\\{\}]
[Mortgage(rate=0.05, duration=30, face\_value=50000, down\_payment=10000.0),
 Mortgage(rate=0.05, duration=30, face\_value=60000, down\_payment=12000.0),
 Mortgage(rate=0.05, duration=30, face\_value=90000, down\_payment=18000.0),
 Mortgage(rate=0.05, duration=30, face\_value=100000, down\_payment=20000.0)]
\end{Verbatim}
\end{tcolorbox}
        
    \begin{tcolorbox}[breakable, size=fbox, boxrule=1pt, pad at break*=1mm,colback=cellbackground, colframe=cellborder]
\prompt{In}{incolor}{90}{\boxspacing}
\begin{Verbatim}[commandchars=\\\{\}]
\PY{c+c1}{\PYZsh{} We can also sort dataclasses using the sorted function. We will sort the mortgage portfolio based on the monthly payment in descending order.}
\PY{n}{sorted\PYZus{}portfolio} \PY{o}{=} \PY{n+nb}{sorted}\PY{p}{(}\PY{n}{mortgage\PYZus{}portfolio}\PY{p}{,} \PY{n}{key}\PY{o}{=}\PY{k}{lambda} \PY{n}{loan}\PY{p}{:} \PY{n}{loan}\PY{o}{.}\PY{n}{monthly\PYZus{}payment}\PY{p}{,} \PY{n}{reverse}\PY{o}{=}\PY{k+kc}{True}\PY{p}{)}
\PY{n}{sorted\PYZus{}portfolio}
\end{Verbatim}
\end{tcolorbox}

            \begin{tcolorbox}[breakable, size=fbox, boxrule=.5pt, pad at break*=1mm, opacityfill=0]
\prompt{Out}{outcolor}{90}{\boxspacing}
\begin{Verbatim}[commandchars=\\\{\}]
[Mortgage(rate=0.05, duration=30, face\_value=100000, down\_payment=20000.0),
 Mortgage(rate=0.05, duration=30, face\_value=90000, down\_payment=18000.0),
 Mortgage(rate=0.05, duration=30, face\_value=60000, down\_payment=12000.0),
 Mortgage(rate=0.05, duration=30, face\_value=50000, down\_payment=10000.0),
 Mortgage(rate=0.05, duration=30, face\_value=10000, down\_payment=2000.0)]
\end{Verbatim}
\end{tcolorbox}
        
    Anything that you can do with python lists, you can do with the list of
dataclasses. For example, we can use the map function to add an
attribute of amortization schedule to each mortgage object in the
portfolio.

Let us define a function that will calculate the amortization schedule
for a mortgage.

    \begin{tcolorbox}[breakable, size=fbox, boxrule=1pt, pad at break*=1mm,colback=cellbackground, colframe=cellborder]
\prompt{In}{incolor}{119}{\boxspacing}
\begin{Verbatim}[commandchars=\\\{\}]
\PY{k}{def} \PY{n+nf}{CalculateAmortizationSchedule}\PY{p}{(}\PY{n}{mortgage}\PY{p}{:} \PY{n}{Mortgage}\PY{p}{)} \PY{o}{\PYZhy{}}\PY{o}{\PYZgt{}} \PY{n+nb}{list}\PY{p}{[}\PY{n+nb}{dict}\PY{p}{]}\PY{p}{:}
  \PY{n}{monthly\PYZus{}rate} \PY{o}{=} \PY{n}{mortgage}\PY{o}{.}\PY{n}{rate} \PY{o}{/} \PY{l+m+mi}{12}
  \PY{n}{monthly\PYZus{}payment} \PY{o}{=} \PY{n}{mortgage}\PY{o}{.}\PY{n}{monthly\PYZus{}payment}
  \PY{n}{balance} \PY{o}{=} \PY{n}{mortgage}\PY{o}{.}\PY{n}{face\PYZus{}value} \PY{o}{\PYZhy{}} \PY{n}{mortgage}\PY{o}{.}\PY{n}{down\PYZus{}payment}
  \PY{n}{amortization\PYZus{}schedule} \PY{o}{=} \PY{p}{[}\PY{p}{]}
  \PY{k}{for} \PY{n}{i} \PY{o+ow}{in} \PY{n+nb}{range}\PY{p}{(}\PY{l+m+mi}{0}\PY{p}{,} \PY{n}{mortgage}\PY{o}{.}\PY{n}{duration} \PY{o}{*} \PY{l+m+mi}{12}\PY{p}{)}\PY{p}{:}
    \PY{n}{interest} \PY{o}{=} \PY{n}{balance} \PY{o}{*} \PY{n}{monthly\PYZus{}rate}
    \PY{n}{principal} \PY{o}{=} \PY{n}{monthly\PYZus{}payment} \PY{o}{\PYZhy{}} \PY{n}{interest}
    \PY{n}{balance} \PY{o}{=} \PY{n}{balance} \PY{o}{\PYZhy{}} \PY{n}{principal}
    \PY{n}{amortization\PYZus{}schedule}\PY{o}{.}\PY{n}{append}\PY{p}{(}\PY{p}{\PYZob{}}\PY{l+s+s2}{\PYZdq{}}\PY{l+s+s2}{period}\PY{l+s+s2}{\PYZdq{}}\PY{p}{:} \PY{n}{i}\PY{o}{+}\PY{l+m+mi}{1}\PY{p}{,} \PY{l+s+s2}{\PYZdq{}}\PY{l+s+s2}{interest}\PY{l+s+s2}{\PYZdq{}}\PY{p}{:}\PY{n}{interest}\PY{p}{,} \PY{l+s+s2}{\PYZdq{}}\PY{l+s+s2}{principal}\PY{l+s+s2}{\PYZdq{}}\PY{p}{:} \PY{n}{principal}\PY{p}{,} \PY{l+s+s2}{\PYZdq{}}\PY{l+s+s2}{unpaid\PYZus{}balance}\PY{l+s+s2}{\PYZdq{}}\PY{p}{:}\PY{n}{balance}\PY{p}{\PYZcb{}}\PY{p}{)}
  \PY{k}{return} \PY{n}{amortization\PYZus{}schedule}
\end{Verbatim}
\end{tcolorbox}

    Now let us use the map function to add the amortization schedule to each
mortgage object in the portfolio.

    \begin{tcolorbox}[breakable, size=fbox, boxrule=1pt, pad at break*=1mm,colback=cellbackground, colframe=cellborder]
\prompt{In}{incolor}{120}{\boxspacing}
\begin{Verbatim}[commandchars=\\\{\}]
\PY{k}{for} \PY{n}{mortgage} \PY{o+ow}{in} \PY{n}{mortgage\PYZus{}portfolio}\PY{p}{:}
  \PY{n}{mortgage}\PY{o}{.}\PY{n}{amortization\PYZus{}schedule} \PY{o}{=} \PY{n}{CalculateAmortizationSchedule}\PY{p}{(}\PY{n}{mortgage}\PY{p}{)}
\PY{n}{mortgage\PYZus{}portfolio}\PY{p}{[}\PY{l+m+mi}{0}\PY{p}{]}\PY{o}{.}\PY{n}{amortization\PYZus{}schedule}\PY{p}{[}\PY{l+m+mi}{0}\PY{p}{:}\PY{l+m+mi}{4}\PY{p}{]} \PY{c+c1}{\PYZsh{} We will print the first four months of the amortization schedule for the first mortgage in the portfolio.}
\end{Verbatim}
\end{tcolorbox}

            \begin{tcolorbox}[breakable, size=fbox, boxrule=.5pt, pad at break*=1mm, opacityfill=0]
\prompt{Out}{outcolor}{120}{\boxspacing}
\begin{Verbatim}[commandchars=\\\{\}]
[\{'period': 1,
  'interest': 33.333333333333336,
  'principal': 9.612396507637854,
  'unpaid\_balance': 7990.387603492362\},
 \{'period': 2,
  'interest': 33.293281681218176,
  'principal': 9.652448159753014,
  'unpaid\_balance': 7980.73515533261\},
 \{'period': 3,
  'interest': 33.25306314721921,
  'principal': 9.692666693751981,
  'unpaid\_balance': 7971.042488638858\},
 \{'period': 4,
  'interest': 33.21267703599524,
  'principal': 9.733052804975948,
  'unpaid\_balance': 7961.309435833882\}]
\end{Verbatim}
\end{tcolorbox}
        
    \begin{tcolorbox}[breakable, size=fbox, boxrule=1pt, pad at break*=1mm,colback=cellbackground, colframe=cellborder]
\prompt{In}{incolor}{121}{\boxspacing}
\begin{Verbatim}[commandchars=\\\{\}]
\PY{k+kn}{import} \PY{n+nn}{matplotlib}\PY{n+nn}{.}\PY{n+nn}{pyplot} \PY{k}{as} \PY{n+nn}{plt}
\PY{n}{schedule\PYZus{}plots} \PY{o}{=} \PY{p}{[}\PY{n}{plt}\PY{o}{.}\PY{n}{plot}\PY{p}{(}\PY{p}{[}\PY{n}{schedule}\PY{p}{[}\PY{l+s+s2}{\PYZdq{}}\PY{l+s+s2}{period}\PY{l+s+s2}{\PYZdq{}}\PY{p}{]} \PY{k}{for} \PY{n}{schedule} \PY{o+ow}{in} \PY{n}{mortgage}\PY{o}{.}\PY{n}{amortization\PYZus{}schedule}\PY{p}{]}\PY{p}{,} \PY{p}{[}\PY{n}{schedule}\PY{p}{[}\PY{l+s+s2}{\PYZdq{}}\PY{l+s+s2}{unpaid\PYZus{}balance}\PY{l+s+s2}{\PYZdq{}}\PY{p}{]} \PY{k}{for} \PY{n}{schedule} \PY{o+ow}{in} \PY{n}{mortgage}\PY{o}{.}\PY{n}{amortization\PYZus{}schedule}\PY{p}{]}\PY{p}{,} \PY{n}{label}\PY{o}{=}\PY{l+s+sa}{f}\PY{l+s+s2}{\PYZdq{}}\PY{l+s+s2}{Loan }\PY{l+s+si}{\PYZob{}}\PY{n}{i}\PY{o}{+}\PY{l+m+mi}{1}\PY{l+s+si}{\PYZcb{}}\PY{l+s+s2}{\PYZdq{}}\PY{p}{)} \PY{k}{for} \PY{n}{i}\PY{p}{,} \PY{n}{mortgage} \PY{o+ow}{in} \PY{n+nb}{enumerate}\PY{p}{(}\PY{n}{mortgage\PYZus{}portfolio}\PY{p}{)}\PY{p}{]}
\PY{n}{plt}\PY{o}{.}\PY{n}{title}\PY{p}{(}\PY{l+s+s2}{\PYZdq{}}\PY{l+s+s2}{Amortization Schedule for all the loans in the portfolio}\PY{l+s+s2}{\PYZdq{}}\PY{p}{)}
\PY{n}{plt}\PY{o}{.}\PY{n}{xlabel}\PY{p}{(}\PY{l+s+s2}{\PYZdq{}}\PY{l+s+s2}{Months}\PY{l+s+s2}{\PYZdq{}}\PY{p}{)}
\PY{n}{plt}\PY{o}{.}\PY{n}{ylabel}\PY{p}{(}\PY{l+s+s2}{\PYZdq{}}\PY{l+s+s2}{Unpaid Balance}\PY{l+s+s2}{\PYZdq{}}\PY{p}{)}
\PY{n}{plt}\PY{o}{.}\PY{n}{legend}\PY{p}{(}\PY{p}{)}
\PY{n}{plt}\PY{o}{.}\PY{n}{show}\PY{p}{(}\PY{p}{)}
\end{Verbatim}
\end{tcolorbox}

    \begin{center}
    \adjustimage{max size={0.9\linewidth}{0.9\paperheight}}{Dataclasses_files/Dataclasses_28_0.png}
    \end{center}
    { \hspace*{\fill} \\}
    
    \hypertarget{some-disadvantages}{%
\subsection{Some disadvantages}\label{some-disadvantages}}

There are multiple ways to achieve the same thing in Python. Dataclasses
are useful but are not a substitute for pandas dataframes. Dataclasses
do not have many out-of-the-box features as pandas dataframes provide
such as rich support for date-time libraries and plotting. If the data
makes more sense in a tabular format, then pandas dataframes are the way
to go. For example, if you need time-series features like moving
averages, differencing, etc., then do not use dataclasses, unless you
are willing to write the code to implement these features.

However, dataclasses are useful when we want to create a simple data
structure that is easy to understand and maintain.

    


    % Add a bibliography block to the postdoc
    
    
    
\end{document}
